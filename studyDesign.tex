%\section{study design}
\noindent 
The main goal of the study is to provide language designers with a feature model as an instrument that helps in designing the most suitable language and its associated environment depending on needs at hand. To solve this problem we extract key features that differentiate mission specification environments; with the view of understanding the language features and mission specification concepts implemented in these environments. We identified the following research questions.\\
\noindent
\textbf{RQ1}: What mission specification environments exist?\\
\claudio{This RQ is impossible to answer and it does not make sense.}
This question identifies mission specification environments that are available in the public domain.
%\claudio{Are we identifying all the environments? the most prominent? what does most prominent mean? How do we measure most prominent?} 
These include open source environments and those available for research and academic purposes. The environments should be end-user facing.
\claudio{I think that we should mention that to answer RQ2 we collected mission specification environments (as part of the methodology, but RQ1 per se has no sense. It is obvious that we cannot collect ``all the possible" mission specification environments.}\\
\textbf{RQ2}: What language features define and differentiate the specification environments for end-user programming?\\
This question identifies the language features in the specification environments and provides the opportunity to see what makes one language different from others.\\
%\textbf{RQ3}: How mission specification environments differ?\\
%By consideration features in the respective environments we seek to identify how environment differs among each other. 
%This will  help users understanding the most suitable mission specification environment depending on their needs.\\ 
%\claudio{Notation. choose among end-user and end-user and mission specification and mission specification and be consistent}
