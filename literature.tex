\section{Related Work}

\parhead{Related Surveys.}
Bravo et al.\,\cite{Bravo2018} in their review of intuitive robot programming environments for education categorized robot programming languages into textual, visual, and tangible languages. However, they did not discuss individual language features that facilitate end-user programming, as we do.% The survey gave us the basis to figure out what identifies the various visual, textual and tangible programming languages as shown in figure \ref{fig:featuremodel}.
Biggs et al.\,\cite{Biggs2003} surveyed robot programming systems, in which they have classified the systems into manual programming systems and automatic systems. The manual ones require users to specify missions, while automatic systems control robots based on their interactiosn with the environment, indicating that such missions are specified on a higher level, for instance, by declaring the mission goals instead of the concrete movements. However, the survey did not discuss language features that enhance robot programming by novice programmers.
Celine Ray et al.\,\cite{CelineRayFrancescoMondadaMemberIEEEandRolandSiegwartFellow2008} in their survey of what people expect from robots realized that elements that appear at personal level include; household daily tasks, security, entertainment and company (child, animal, elderly care). More than half of them expect the robots providing such services to be in market soon. For someone to get such services from robots, it becomes necessary that mission specification be raised to higher level of expressiveness which is close to the user domain for easy management of several missions. However, the survey did not cover the available languages, which promote flexible programming of robots by end-user and identify the features required to support novice programmers.
Abdelfetah Hentout et al.\,\cite{Hentout2017} survey development environments for robotics. They identified frameworks for programming robotic system, but not targeting mission specification.
%In ensuring high degree of usability of robotic systems, it is important that the primitives used to specify missions depict the aspects of that user domain \cite{Benyon1996}, while considering the mission behavior to robot capabilities.

% Block programming: Ability to program robots requires enormous effort and time making it hard to deploy them in the community \cite{Weintrop2018}\tb{what kind of work is that?}. New robots are more powerful and flexible, however in order to support robot presence in everyday lives, there is need for quick reprogramming support. Lego mindstroms uses visual blocks to represent basic robot actions which users can organize to produce desired outcomes. There is a challenge of creating meaningful icon for every possible command. Sample tools include;(a)Universal Robot’s tree-based programming tool, in which specification is done  both graphically and and text script, ... (b) Lego mindstorms ...

A survey on DSLs for robotics by Nordmann et al.\,\cite{Nordmann2016a} identified a large number of languages, but surprisingly none of the languages supports mission specification, making it distinct from our study.
%analyzed various issues about modelling languages, with negligible work on mission specification in particular.
Specifically, the survey covered aspects of environmental features and constraints, which are expressed using formalism such as: LTL, OWL, and (E)BNF. While scenario definitions are made using formalisms such as ANTLR grammar, (E)BNF, UML/MOF, LTL or Ecore, all these formalims are best used by robotic and software engineers, but not end users. This gap also motivates our study.

\begin{itemize}
	\item \citet{bacca.ea:2017:teachingrobots} \tb{Swaib, can you check this one? they list a couple more environments, some of which aren't covered in our survey (actually, you have this publication in the repo, as a material for Edison; did you check the environments they listed in addition to ours?)}
	\item another argument over the existing surveys: we provide the most complete one, relying on a systematic search process
	\item 
\end{itemize}


\parhead{Beyond Language-Based Mission Specification}
Gorostiza et al.\,\cite{Gorostiza2011} propose a natural programming environment in which robot skills are accessed verbally to interact with the end-user. The environment uses dialog system to extract actions and conditions to create a sequence function chart. The challenge with this approach is, the end-user can not add new dialogue construct for new tasks, making the languages inflexible to end-user.
Miguel Campusano et al.\,\cite{campusano.ea:2017:live} in their work on ``live robot programming'' have implemented a language which supports live  feedback, which helps the end-user in rapid creation and variation of robot behavior at run-time. This approach however does not provide end-user domain constructs to simplify the programming effort during mission specification. Doherty et al \cite{Doherty2013}, propose a framework and architecture for automated specification, generation and execution missions for multiple drones, which collaborate with humans. The focus of the study is on how the language can clearly and concisely specify and generate mission, but not the ease of using the language by end-users.
%\parhead{Other works}
% Rutle et al. \cite{rutle.ea:2018:commonlang} present a DSL for specifying robotic tasks\tb{not sure how this is related work}
