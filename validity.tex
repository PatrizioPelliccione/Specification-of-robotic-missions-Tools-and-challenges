\section{Threats to Validity}
%\tb{systematically write up threats to external, internal, and conclusion validity; and how the threats were mitigated}\\
\parheadit{Threads to internal validity}. Since observational techniques require considerable knowledge to interpret correctly, the features identified are to the best of the teams knowledge of the domains covered. This effect has been mitigated by distributing environments among authors to collecting features and allowing one author to verify the features collected by another, followed by discussions to reconcile any differences on views. This made the data collection rigorous and thorough. Secondly, Google search engine returns different results to different people on the same search due to personalized search behaviour customized by Google. Therefore if anyone else did the same search the results would not necessarily be the same. This has been resolved by relying on multiple sources of data; using robot names for searching and snowballing.\\
\parheadit{Threads to external validity}. A key thread to external validity is extracting the feature using independent data collection based on documentation available in public domain. If we contacted the developers of the systems to provide more information on some technical aspects. More features would be discovered, more especially for commercial environments like \picaxe, \edison and \blocklyprop. However this has been countered by open source environments with git repositories having all the documentations requires to understand the features of robotic language design space, such as \openroberta, \trik, and \aseba. 
Secondly, diversity in terms used to name mission specification. We have tried to include as much as possible the common phrases that refer to robot mission specification as seen in our search string: "\emph{("programmable robots" OR ("robot programming" OR "mission specification") environment) "mobile robot"} ".\\
%\parheadit{Threads to conclusion validity} \swaib{TODO after the conclusion is written}
%$\bullet$ The features identified are based on the documentations available and what users can observe while using the environment. There could be other features that would be extracted with direct interaction with environment vendors and/or developers.