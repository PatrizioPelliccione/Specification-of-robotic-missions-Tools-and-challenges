\section{Background}
A typical robot mission is specified by a user by programming a behavior, which the robot exhibits during execution. The mission specified is either compiled into robot executable sequence of instruction or directly interpreted depending on the semantic support. During deployment the executable file is transferred to the robot either by cable connection or wireless. Environments which support live programming enable users to specify such missions at run-time. Visual programming languages based on libraries like blockly \cite{blockly} and scratch are used to define language constructs which express robotic and end-user domain mission primitives like move forward/backwards, read sensor, turn right/left as seen in \openroberta, \picaxe, \edison and \sphero. Below are some terminologies used:\\% \tb{can you briefly describe the process of mission specification here and how environments look like; for instance, explain the typical libraries, Blockly and Scratch (see below) that are used to realize domain-specific languages}
$\bullet$ \emph{End-user facing:} an environment that is target for being used by non expert end-users.\\ %\deleted[id=CM]{The environment provides concrete syntax to end-users for specifying tasks.}\\
$\bullet$ \emph{Specification of Robotic mission:} is the definition of the goal of the robotic application or the tasks that must be performed by the robot(s). \\
$\bullet$ \emph{Specification  Environment:} is a collection of language tools and features with abstraction support for robotics and end-users, suitable for the definition and execution of robotic missions.\\
$\bullet$ \emph{Mobile Robots:} robots with ability to change  their locations in space like: drones, mobile manipulators, mobile platforms, mobile humanoids, vacuum cleaner robots.\\
$\bullet$ \emph{Declarative vs imperative mission:} a declarative mission allows end-users to specify what to be achieved i.e., a mission goal without saying how the goal is achieved. While an  imperative mission allows End-users to specify the explicit set of tasks in their order of execution.// %Imperative  mission specification  are usually specified by user domain experts who exactly know how the mission tasks are executed.\\
$\bullet$ \emph{Block:} a graphical symbol or icon  which represents  a language construct.  Blocks are connected to build a mission. For example "moves steering" block in LEGO, "if do else" block in Open Roberta ~\cite{OpenRoberta}\\ %\claudio{Remove (comment)}
$\bullet$ \emph{Blockly:} is a library by Google, which provides support for creating graphical notations, where each block represents a programming concept. %The library is used for building web based and mobile applications with graphical notations.\\ https://developers.google.com/blockly/guides/overview %\claudio{Remove Blockly (comment)}
%Common examples we captured in this study include Google
Blockly\,\cite{blockly}, blocks enforce semantic correctness of the program by allowing two blocks to connect only if they make semantic sense.
%Code can be generated as; python, JavaScript or PHP.
The library can be extended to define new blocks, support functions, and procedures. Blockly allows access to the parse tree, and is flexible to generate code in target language\,\cite{Passault2016}. Scratch is a variant of Blockly created by MIT, which can be extended by adding custom blocks to suit needs of the End-user domain.

\parhead{Educational Robotics}
Mobile robots have been recognized as an ideal vehicle to teach basic programming skills to children and students. As such, a large variety of educational robots has been proposed, such as Lego Mindstorms, Softbank Robotics' NAO or Pepper
%\tb{also explain scratch as a library, similar to blockly}