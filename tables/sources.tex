\begin{table*}[t]
\caption{Identified environments per step and data source (cf. \secref{sec:sel})}
\label{tab:DataSources}
\vspace{-.4cm}
\begin{threeparttable}
\begin{tabular}{p{7cm}p{10cm}}\\
\toprule
\textsf{data source} & \textsf{identified environments}\\
\midrule
Environments from experience (25 candidates) & \missionlab, \choregraphe, \lego, \sphero, \tivipe, \aseba, \robotmesh, \edison, \makeblock, \trik, \ardublockly, \minibloq, and \flyaq. \\
List of mobile robots (59 candidates) & \picaxe, \openroberta, \arcbotics, \vex, \metabot, \marty, \tello, and \codelab. \\
Google search (373 candidates) & \missionlab, \choregraphe, \lego, \sphero, \tivipe, \aseba, \robotmesh, \edison, \makeblock, \trik, \ardublockly, \minibloq, \flyaq, \picaxe, \openroberta, \arcbotics, \vex, \metabot, \marty, \tello, \codelab,  \blocklyprop, and \ozoblockly.  \\  % -> what did "\codelab. unique" mean?
Snowballing (80 candidates) &  \lego, \missionlab, \aseba, \vex, \choregraphe, \minibloq,  \ozoblockly, \sphero, \tivipe, \openroberta, \trik, \robotmesh, \enchanting, \easyc and \robotc\\ % -> what did "\robotmesh unique" mean?
Further alternative environments\tnote{1}& \turtlebot, \makeblock, \scratchev \\
\bottomrule
\end{tabular}
\begin{tablenotes}
\item[1] Found by seeking alternative environments for robots supported by the identified environments above
\end{tablenotes}
\end{threeparttable} 
\end{table*}
%13 from the team:\missionlab, \choregraphe, \lego, \sphero, \tivipe, \aseba, \robotmesh, \edison, \makeblock, \trik, \ardublockly, \minibloq and \flyaq.  8 environments from mobile robot list: \picaxe, \openroberta, \arcbotics, \vex, \metabot, \marty, \tello, and \codelab.   Interestingly all the 21 from stage 1 and stage 2 were included in stage 3 plus 2 more: \blocklyprop, and \ozoblockly.  Stage 4 generated 15 environments, out of which 12 already exist in stage 3: the exclusive 3 are: \enchanting, \easyc and \robotc.  From alternative environments for robots, 3 selected are: \turtlebot, \makecode, and \scratchev. 