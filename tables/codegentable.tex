\begin{table*}
\caption{Code generation matrix indicating languages of generated code from the graphical notations\tb{remove the classification according to the notation, just have a two-column table: first column: target language, second column: environment generating it; use the other tables' style}}
\label{Codegeneration}
\begin{tabular}{ |m{4em}|m{3cm}|m{2cm}|m{2cm}|m{3cm}|m{3cm}|}
\hline
\textbf{Graphical Notation} &\textbf{C/C++} &\textbf{Java} &\textbf{Java Script} &\textbf{Python}& \textbf{Others}\\
\hline
Blockly & \robotc, \blocklyprop, \robotmesh, \arcbotics, \openroberta  &\openroberta  & \openroberta, \makecode, \ozoblockly &\openroberta, \turtlebot, \robotmesh & \ardublockly (Arduino code), makebot (assembly code), \\
\hline
Scratch & &\enchanting, \scratchev, \vex & \sphero   & \tello, \makeblock, \marty & \\
\hline
State machine & \trik, \choregraphe, \missionlab &   &  \trik ,\choregraphe& \trik \choregraphe& \trik(F\#, PascalABC, NXT OSEK C), \choregraphe (Matlab, .NET), \picaxe(Basic) \\
\hline
Custom & \easyc, \minibloq, \tivipe & & &\edison &\flyaq(QBL), \aseba (VPL to Aseba event scripting language AESL) \\
\hline

\end{tabular}
\end{table*}

